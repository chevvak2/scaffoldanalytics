
%%%%%%%%%%%%%%%%%%%%%%%%%%%%%%%%%%%%%%%%%%%%%%%%%%%%%%%%%%%%%%%%%%%%%
%% This is a (brief) model paper using the achemso class
%% The document class accepts keyval options, which should include
%% the target journal and optionally the manuscript type.
%%%%%%%%%%%%%%%%%%%%%%%%%%%%%%%%%%%%%%%%%%%%%%%%%%%%%%%%%%%%%%%%%%%%%
\documentclass[journal=jacsat,manuscript=article]{achemso}

%%%%%%%%%%%%%%%%%%%%%%%%%%%%%%%%%%%%%%%%%%%%%%%%%%%%%%%%%%%%%%%%%%%%%
%% Place any additional packages needed here.  Only include packages
%% which are essential, to avoid problems later. Do NOT use any
%% packages which require e-TeX (for example etoolbox): the e-TeX
%% extensions are not currently available on the ACS conversion
%% servers.
%%%%%%%%%%%%%%%%%%%%%%%%%%%%%%%%%%%%%%%%%%%%%%%%%%%%%%%%%%%%%%%%%%%%%
%%\usepackage[version=3]{mhchem} % Formula subscripts using \ce{}
\usepackage[T1]{fontenc}       % Use modern font encodings

%%%%%%%%%%%%%%%%%%%%%%%%%%%%%%%%%%%%%%%%%%%%%%%%%%%%%%%%%%%%%%%%%%%%%
%% If issues arise when submitting your manuscript, you may want to
%% un-comment the next line.  This provides information on the
%% version of every file you have used.
%%%%%%%%%%%%%%%%%%%%%%%%%%%%%%%%%%%%%%%%%%%%%%%%%%%%%%%%%%%%%%%%%%%%%
%%\listfiles

%%%%%%%%%%%%%%%%%%%%%%%%%%%%%%%%%%%%%%%%%%%%%%%%%%%%%%%%%%%%%%%%%%%%%
%% Place any additional macros here.  Please use \newcommand* where
%% possible, and avoid layout-changing macros (which are not used
%% when typesetting).
%%%%%%%%%%%%%%%%%%%%%%%%%%%%%%%%%%%%%%%%%%%%%%%%%%%%%%%%%%%%%%%%%%%%%
\newcommand*\mycommand[1]{\texttt{\emph{#1}}}

%%%%%%%%%%%%%%%%%%%%%%%%%%%%%%%%%%%%%%%%%%%%%%%%%%%%%%%%%%%%%%%%%%%%%
%% Meta-data block
%% ---------------
%% Each author should be given as a separate \author command.
%%
%% Corresponding authors should have an e-mail given after the author
%% name as an \email command. Phone and fax numbers can be given
%% using \phone and \fax, respectively; this information is optional.
%%
%% The affiliation of authors is given after the authors; each
%% \affiliation command applies to all preceding authors not already
%% assigned an affiliation.
%%
%% The affiliation takes an option argument for the short name.  This
%% will typically be something like "University of Somewhere".
%%
%% The \altaffiliation macro should be used for new address, etc.
%% On the other hand, \alsoaffiliation is used on a per author basis
%% when authors are associated with multiple institutions.
%%%%%%%%%%%%%%%%%%%%%%%%%%%%%%%%%%%%%%%%%%%%%%%%%%%%%%%%%%%%%%%%%%%%%
\author{Deepak Bandyopadhyay}
\email{deepak.2.bandyopadhyay@gsk.com}
\author{Constantine Kreatsoulas}
\author{Pat G. Brady}
\author{Joseph Boyer}
\author{Zangdong He}
\author{Genaro Scavello Jr.}
\affiliation[GSK]{GlaxoSmithKline, 1250 S. Collegeville Rd, Collegeville, PA 19426}
\author{Dac-Trung Nguyen}
\author{Tyler Peryea}
\author{Rajarshi Guha}
\email{guhar@mail.nih.gov}
\author{Ajit Jadhav}
\email{ajadhav@mail.nih.gov}
%\phone{+123 (0)123 4445556}
%\fax{+123 (0)123 4445557}
\affiliation[NCATS]{National Center for Advancing Translational Science, 9800 Medical Center Drive, Rockville, MD 20850}

%%%%%%%%%%%%%%%%%%%%%%%%%%%%%%%%%%%%%%%%%%%%%%%%%%%%%%%%%%%%%%%%%%%%%
%% The document title should be given as usual. Some journals require
%% a running title from the author: this should be supplied as an
%% optional argument to \title.
%%%%%%%%%%%%%%%%%%%%%%%%%%%%%%%%%%%%%%%%%%%%%%%%%%%%%%%%%%%%%%%%%%%%%
\title[Scaffold Analytics] {Scaffold-Based Analytics: Enabling Hit-to-Lead
  Decisions by Visualizing Chemical Series Linked Across Large Datasets}  

%%%%%%%%%%%%%%%%%%%%%%%%%%%%%%%%%%%%%%%%%%%%%%%%%%%%%%%%%%%%%%%%%%%%%
%% Some journals require a list of abbreviations or keywords to be
%% supplied. These should be set up here, and will be printed after
%% the title and author information, if needed.
%%%%%%%%%%%%%%%%%%%%%%%%%%%%%%%%%%%%%%%%%%%%%%%%%%%%%%%%%%%%%%%%%%%%%
%\abbreviations{IR,NMR,UV}
%\keywords{American Chemical Society, \LaTeX}

%%%%%%%%%%%%%%%%%%%%%%%%%%%%%%%%%%%%%%%%%%%%%%%%%%%%%%%%%%%%%%%%%%%%%
%% The manuscript does not need to include \maketitle, which is
%% executed automatically.
%%%%%%%%%%%%%%%%%%%%%%%%%%%%%%%%%%%%%%%%%%%%%%%%%%%%%%%%%%%%%%%%%%%%%
\begin{document}

%%%%%%%%%%%%%%%%%%%%%%%%%%%%%%%%%%%%%%%%%%%%%%%%%%%%%%%%%%%%%%%%%%%%%
%% The "tocentry" environment can be used to create an entry for the
%% graphical table of contents. It is given here as some journals
%% require that it is printed as part of the abstract page. It will
%% be automatically moved as appropriate.
%%%%%%%%%%%%%%%%%%%%%%%%%%%%%%%%%%%%%%%%%%%%%%%%%%%%%%%%%%%%%%%%%%%%%
\begin{tocentry}

Some journals require a graphical entry for the Table of Contents.
This should be laid out ``print ready'' so that the sizing of the
text is correct.

Inside the \texttt{tocentry} environment, the font used is Helvetica
8\,pt, as required by \emph{Journal of the American Chemical
Society}.

The surrounding frame is 9\,cm by 3.5\,cm, which is the maximum
permitted for  \emph{Journal of the American Chemical Society}
graphical table of content entries. The box will not resize if the
content is too big: instead it will overflow the edge of the box.

This box and the associated title will always be printed on a
separate page at the end of the document.

\end{tocentry}

%%%%%%%%%%%%%%%%%%%%%%%%%%%%%%%%%%%%%%%%%%%%%%%%%%%%%%%%%%%%%%%%%%%%%
%% The abstract environment will automatically gobble the contents
%% if an abstract is not used by the target journal.
%%%%%%%%%%%%%%%%%%%%%%%%%%%%%%%%%%%%%%%%%%%%%%%%%%%%%%%%%%%%%%%%%%%%%
\begin{abstract}
  We present a method for visualizing and navigating large and diverse chemical spaces, such as screening datasets, along with their activities and properties. Our approach is to annotate the data with all possible scaffolds contained within each molecule using an exhaustive algorithm developed at NCATS.  We have developed a Spotfire visualization that is used to drive the hit triage process. Progression decisions can be made using aggregate scaffold parameters and data from multiple datasets merged at the scaffold level.  This visualization easily reveals overlaps that help prioritize hits, highlight tractable series and posit ways to combine aspects of multiple hits.  The SAR of a large and complex hit is automatically mapped into all constituent scaffolds making it possible to navigate, via any shared scaffold, to all related hits.  This scaffold ``walking'' helps address bias toward a handful of potent and ligand-efficient molecules at the expense of coverage of chemical space. The mapping also automates the laborious process of substructure searches within a dataset as structures are now linked to pre-processed search results.  We compare the NCATS scaffold generation method with published screening triage methods such as nearest-neighbor clustering, data-driven clustering and scaffold networks.  We believe that our Spotfire visualization used in combination with structure annotation provides  a novel view of large and diverse datasets. This allows teams to effortlessly navigate between structurally related molecules and enriches the population of leads considered and progressed in a manner complementary to established approaches.
\end{abstract}

%%%%%%%%%%%%%%%%%%%%%%%%%%%%%%%%%%%%%%%%%%%%%%%%%%%%%%%%%%%%%%%%%%%%%
%% Start the main part of the manuscript here.
%%%%%%%%%%%%%%%%%%%%%%%%%%%%%%%%%%%%%%%%%%%%%%%%%%%%%%%%%%%%%%%%%%%%%
\section{Introduction}

The advantage and disadvantage of high throughput screening (HTS) campaigns is
the large amount of data that is generated. While the value of large scale HTS
has been debated\cite{Macarron:2011qv}, the massive structure-activity
datasets generate create a challenge in identifying truly active compounds and
their analogs and weeding out false positives. The process of reducing HTS
datasets from hundreds of thousands of compounds to a few thousand or few
hundred active series is termed triaging. Over the last twenty years many
approaches to HTS triaging have been described which include activity based
thresholds [REF], similarity to known actives, enrichment based
approaches\cite{Varin2010CSE,Pu:2012wf}, ranges of physicochemical
properties\cite{Cox:2012qy}, crowdsourcing\cite{Peng:2013qp} and removal of
promiscuous or otherwise undesirable chemotypes \cite{Dahlin:2014fp}. See
\citeauthor{Shun:2011sy} and \citeauthor{Langer:2009mw} for a review of HTS
triage approaches.

One of the key challenges in the triage step is to identify structure-activity
series - sets of compounds with similar or analogous structures that exhibit a
spectrum of activity. Ideally, such a collection will also contain a number of
inactive compounds as well. Identifying such subsets allows one to have some
confidence in the presence of a structure-activity relationship amongst the
active compounds which enables a more efficient exploration of the chemical
space around the selected hits.

Given that a SAR series is, ideally, defined in terms of a core structure along
with various decorations, a natural first step in the triage process is to
identify these core structures, termed scaffolds. Usually this starts by
decomposing the structures in the screening collection, either exhaustively or
else using one of the many methods to decompose structures into fragments such
as the Bemis-Murcko\cite{BemisMurcko1999,BemisMurcko1996} or RECAP
methods\cite{Lewell:1998aa}. These methods lead to a large number of
fragments ranging from trivial ones such as a benzene ring to very complex
multiring structures. Thus a key step involves identifying the relevant set of
scaffolds and their associated compounds. This can be challenging since a given
compound can contain multiple scaffolds.

In this work we present a HTS triage workflow based on navigating the
scaffold-activity landscape of a screening collection. The workflow includes
methods to visualize the activity landscape as well as methods to explore
different regions of chemical space via shared scaffolds. We also highlight the
efficiency gains obtained by using pre-computed scaffold associations rather
than performing substructure searches. \textbf{Is there a scaffold
  ranking/prioritization procedure that should/could be described?}

\subsection{Related Work}
As noted above there are many ways to generate scaffolds. A key is to identify a
relevant subset of them or else aggregate them in a way that leads to a useful
clustering of active and inactive compounds. While the term ``useful'' is rather
subjective, it is easy to identify cases that are \emph{not} useful. Thus, 5- or
6-member undecorated rings are likely not useful since they will occur in the
majority of compounds in a screening collection. At the other extreme, large,
extended scaffolds that are associated with very few compounds are also likely
not useful.

As a result, many approaches to scaffold aggregation have been described. A
natural approach is to consider a hierarchical aggregation. The Scaffold
Tree\cite{Ertl2011ScaffoldTree} and Scaffold Network\cite{Varin2011ScafNet}
define a hierarchical decomposition from more specialized larger scaffolds to
more inclusive smaller scaffolds. While the Scaffold Tree splits each larger
scaffold in exactly one way into two scaffolds with fewer rings, the Scaffold
Network performs an exhaustive decomposition into all possible smaller scaffolds
with fewer rings.  Since some subscaffolds are shared with neighboring
scaffolds, this produces a network or graph rather than a tree. The Scaffold
Network Generator program\cite{Matlock2013SNG} is a fast, easy-to-use public
command line implementation of Scaffold Networks that can scale to millions of
compounds via aggregation of scaffolds from split datasets and parallel
execution.  \citeauthor{Harper2004DDclus} use exhaustive enumeration to find all
Bemis-Murcko like frameworks in each molecule, and then recursively retain those
frameworks with highest aggregate activity and remove molecules that contain
them until a threshold is met, yielding a set of disjoint frameworks.  Other
methods have used multiple common substructure (MCS), first proposed for finding
protein structural similarity\cite{Koch1997MCSprot}, for example
\cite{Quintus2009MCS} and the ChemAxon product LibraryMCS.

To overcome the effect of small variations in heteroatoms (eg. O to S) mapping
otherwise similar molecules to different scaffolds, generalized or consolidated
scaffold representations have been proposed, such as the Reduced
Graph\cite{Barker2003RG}, the Bemis-Murcko scaffold\cite{BemisMurcko1996} and
the topological or 2D pharmacophore\cite{Schneider1999ScafHopTP}. Bemis-Murcko
like frameworks \cite{Harper2004DDclus} are generalized Bemis-Murcko scaffolds
where atom types and/or bond orders may be retained.

Multiple scaffolds if present in a dataset can be inferred from the scaffold
tree decomposition\cite{ClarkLabute2008SAReport}. However in practice, the
thresholds used by \citeauthor{ClarkLabute2008SAReport} miss common scaffolds in
HTS-like diverse chemical compound sets. \citeauthor{Bandyopadhyay2011ACS} have
used a common fragment decomposition plug-in for SAReport in order to export
scaffolds for diverse datasets.

\section{Methods}
\label{sec:methods}

\begin{itemize}
\item Scaffold decomposition algorithm, description
\item Linking R-group tool to Spotfire
\item Developing the Spotfire vis UI
\end{itemize}

\subsection{Datasets}
\label{sec:datasets}

\section{Results}

\section{Discussion}
\label{sec:discussion}

\begin{itemize}
\item Discuss performance - scaffold generation is usually a one time procedure
  for a given screening deck. Furthermore, the scaffold generating process will
  associate compounds with scaffolds, so looking up scaffold membership is very
  fast. When considering scaffold similarity, usual performance bottlenecks
  occur, same as for other similarity applications. Can we include some
  performance numbers?
\item we haven't discussed removal of promiscuous compounds/chemotypes,
  undesirable chemotypes (ie PAINS etc), these are generally a separate and
  independent step from the scaffold identification process
\item Ranking scaffolds is a key step in prioritizing hits in a scaffold-based
  approach. Still a subjective issue and many ways to do it. Not clear that
  there is a single optimal way
\item Privileged scaffolds - sometimes could go in looking for them, but usually
  one identifies such privileged scaffolds in a retrospective fashion, across
  multiple HTS campaigns. Relevance / importance to scaffold based triage?
\end{itemize}

\begin{acknowledgement}
  The GSK authors thank Subhas Chakravorty, Neysa Nevins, Ami Lakdawala Shah,
  Eric Manas, Todd Graybill, Stan Martens, Mike Ouellette, Tony Jurewicz and Rob
  Young for valuable feedback and suggestions while developing the method and
  visualizations.
\end{acknowledgement}

%%%%%%%%%%%%%%%%%%%%%%%%%%%%%%%%%%%%%%%%%%%%%%%%%%%%%%%%%%%%%%%%%%%%%
%% The same is true for Supporting Information, which should use the
%% suppinfo environment.
%%%%%%%%%%%%%%%%%%%%%%%%%%%%%%%%%%%%%%%%%%%%%%%%%%%%%%%%%%%%%%%%%%%%%
\begin{suppinfo}
Supplementary material is available online for this article.
\end{suppinfo}

%%%%%%%%%%%%%%%%%%%%%%%%%%%%%%%%%%%%%%%%%%%%%%%%%%%%%%%%%%%%%%%%%%%%%
%% The appropriate \bibliography command should be placed here.
%% Notice that the class file automatically sets \bibliographystyle
%% and also names the section correctly.
%%%%%%%%%%%%%%%%%%%%%%%%%%%%%%%%%%%%%%%%%%%%%%%%%%%%%%%%%%%%%%%%%%%%%
\bibliography{bibliography}

\end{document}
