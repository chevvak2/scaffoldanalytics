\documentclass[11pt,letterpaper]{article}
\usepackage{ctable}

\renewcommand{\thetable}{S\arabic{table}}
\renewcommand{\thesection}{S\arabic{section}}

\setlength{\parindent}{0em}
\setlength{\parskip}{1em}

\begin{document}
\title{Scaffold-Based Analytics: Enabling Hit-to-Lead Decisions by
  Visualizing Chemical Series Linked Across Large Datasets}
\author{Deepak Bandyopadhyay$^{\dagger}$, Constantine Kreatsoulas$^{\dagger}$
  Pat G. Brady$^{\dagger}$, \\
  Joseph Boyer$^{\dagger}$, Zangdong He$^{\dagger}$, 
  Genaro Scavello Jr.$^{\dagger}$,\\
  Dac-Trung Nguyen$^{\ddagger}$,
  Tyler Peryea$^{\ddagger}$,
  Rajarshi Guha$^{\ddagger}$,
  Ajit Jadhav$^{\ddagger}$ \\
\\
$^{\dagger}$ GlaxoSmithKline, 1250 S. Collegeville Rd, Collegeville,
PA 19426 \\
\\
$^{\ddagger}$ National Center for Advancing Translational Science, \\ 9800 Medical Center Drive, Rockville, MD 20850}
\maketitle

\newpage 

\section{NCATS R-group tool output files}
\label{sec:ncats-r-group}

\begin{table}[h]
  \centering
  \begin{tabular}[h]{lp{0.75\linewidth}}
    \hline
    \textbf{Column Name} & \textbf{Description} \\
    \hline
    ScaffoldID & Numeric scaffold identifier. Each scaffold occurs only
    once, and data columns are aggregated for all molecules containing the
    scaffold \\
    Structure & Scaffold SMILES without R-groups attached \\
    RgroupLabels & A comma separated list of R-group labels for
    all R-groups associated with the scaffold \\
    ScaffoldScore & A quantitative assessment of the scaffold quality.
    See Section \ref{sec:scaffold-metrics} \\
    Complexity & A number that captures increasing size and complexity of
    scaffolds. See Section \ref{sec:scaffold-metrics}.  \\
    Count & Number of molecules that share this scaffold \\
    \hline
  \end{tabular}
  \caption{A description of the fixed columns of the scaffold file generated
    by the NCATS R-group tool. Additional columns may be present which
    correspond to aggregated property columns. Thus for each
    property of the input molecules, we compute the mean and standard
    deviation of that property for all molecules containing the
    scaffold. These values are reported in columns labeled \texttt{X} and
    \texttt{$X_{sd}$}, where \texttt{X} is the property name.}
  \label{table:scaffoldfilecolumns}
\end{table}

\begin{table}[h]
  \centering
  \begin{tabular}[h]{lp{0.75\linewidth}}
    \hline
    \textbf{Column Name} & \textbf{Description} \\
    \hline
    ScaffoldID & Numeric scaffold identifier (corresponding to
  the {\it ScaffoldID} column in the scaffold file, Table \ref{table:scaffoldfilecolumns}) \\
  MolID & Numeric or text molecule identifier (name). Each molecule is
  repeated once for each scaffold that it occurs in \\
  Structure & Molecule structure in SMILES format \\
  $R_1, \ldots, R_n$ & R-group SMILES, with \*-atoms at attachment
  points. By default we limit to $n = 21$ \\
  \hline
  \end{tabular}
  \caption{A description of the columns in the R-group decomposition
    file generated by the NCATS R-group tool.}
\end{table}

\newpage

\section{Scaffold metrics}
\label{sec:scaffold-metrics}

The NCATS R-group tool is designed to fragment a collection of
molecules. In addition to the fragmentation procedure it computes a
series of scaffold metrics, described in Table
\ref{table:scaffoldfilecolumns}. In this section we provide some
details about the \textit{ScaffoldScore} and \textit{Complexity}
metrics.

The \textit{ScaffoldScore} is an empirical metric designed to
summarize a scaffold (or more generally, a fragment) and the compounds
containing the scaffold. Specifically, we define it as  
\begin{equation}
  \label{eq:1}
  S = -\log_{10} \Bigg( \sqrt{ N_{\mathrm{core}} \times \frac{N_{m}}{N} \times
    \frac{1}{\sqrt{\sigma}} \times \frac{1}{R} }  \, \Bigg)
\end{equation}
where $N_{\mathrm{core}}$ is the atom count of the scaffold,
$N_{m}$ is the size of the member set for the scaffold,
$N$ is the total number of molecules used as input, $R$ is the number
of R-groups identified for this scaffold and $\sigma$ is a measure of
how close the members are to the scaffold and is defined as
\begin{equation}
  \label{eq:2}
  \sigma = \sum_{i=1}^{N_m} ( A_i - N_{\mathrm{core}} )^2
\end{equation}
where $A_i$ is the atom count of the $i$'th molecule in the scaffolds
member set.  In summary, the score for a scaffold is higher if it is
larger, with fewer R-groups and with member molecules that are
relatively close to the scaffold and cover a large fraction of the
input set.

The \textit{Complexity} metric is an implemention of the empirical
complexity metric described by Barone and Channon \cite{Barone2001}.
\textit{Complexity} can be used to prune away scaffolds that are too
simple, by setting a cutoff such as 100.


\newpage

\bibliographystyle{unsrt}
\bibliography{bibliography}

\end{document}

